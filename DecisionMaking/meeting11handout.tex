\documentclass[12pt,letterpaper]{article}

\usepackage[utf8]{inputenc}
\usepackage[T1]{fontenc}
\usepackage{amsmath}
\usepackage{amsfonts}
\usepackage{amssymb}
\usepackage{amsthm}
\usepackage[left=2cm,right=2cm,top=2cm,bottom=2cm,headheight=22pt]{geometry}
\usepackage{fancyhdr}
\usepackage{setspace}
\usepackage{lastpage}
\usepackage{graphicx}
\usepackage{caption}
\usepackage{subcaption}


\theoremstyle{definition}
\newtheorem{question}{Question}
\newtheorem{example}{Example}
\newtheorem{exercise}[question]{Exercise}
\newtheorem*{challenge}{Challenge}

\begin{document}

%Paramètres de mise en forme des paragraphes selon les normes françaises
\setlength{\parskip}{1ex plus 0.5ex minus 0.2ex}
\setlength{\parindent}{0pt}

%Paramètres relatifs aux en-têtes et pieds de page.
\pagestyle{fancy}
\lhead{Theron J Hitchman}
\chead{\Large Images for Meeting 11}
\rhead{Fall 2013}
\lfoot{\emph{Math and Decision Making}}
\cfoot{}
\rfoot{\emph{\thepage\ of \pageref{LastPage}}}


\section*{Practice with Invariants for Links}

We now have two different invariants for links: \emph{the number of components} and \emph{tricolorability}.
We can use these invariants to distinguish between some types of links.

Below are planar projection diagrams for six links. 


\begin{question} 
Are any of these links distinguishable using our current set of invariants?
If so, say how you know each pair is different.
\end{question}


\begin{question} 
Are any of these links equivalent by ambient isotopy? 
If so, realize this equivalence with a sequence of Reidemeister moves.
\end{question}

\begin{figure}[h]
    \centering
    \begin{subfigure}{.3\textwidth}
        \centering
        \includegraphics[width=\textwidth]{meeting11pics/hopf.png}
        \caption{Link A}
    \end{subfigure}
    \quad
    \begin{subfigure}{.3\textwidth}
        \centering
        \includegraphics[width=\textwidth]{meeting11pics/hopfmirror.png}
        \caption{Link B}
    \end{subfigure}
    \quad
    \begin{subfigure}{.3\textwidth}
        \centering
        \includegraphics[width=\textwidth]{meeting11pics/borromean.png}
        \caption{Link C}
    \end{subfigure}
    
    \begin{subfigure}{.3\textwidth}
        \centering
        \includegraphics[width=\textwidth]{meeting11pics/whitehead.png}
        \caption{Link D}
    \end{subfigure}
    \quad
    \begin{subfigure}{.3\textwidth}
        \centering
        \includegraphics[width=\textwidth]{meeting11pics/borromean-trefoil.png}
        \caption{Link E}
    \end{subfigure}
    \quad
    \begin{subfigure}{.3\textwidth}
        \centering
        \includegraphics[width=\textwidth]{meeting11pics/unlink.png}
        \caption{Link F}
    \end{subfigure}
    \caption{Some Links}
\end{figure}

\end{document}