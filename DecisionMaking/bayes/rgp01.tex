\documentclass[12pt,letterpaper]{article}

\usepackage[utf8]{inputenc}
\usepackage[T1]{fontenc}
\usepackage{amsmath}
\usepackage{amsfonts}
\usepackage{amssymb}
\usepackage{amsthm}
\usepackage[left=2cm,right=2cm,top=2cm,bottom=2cm,headheight=22pt]{geometry}
\usepackage{fancyhdr}
\usepackage{setspace}
\usepackage{lastpage}
\usepackage{graphicx}
\usepackage{caption}
\usepackage{subcaption}
\usepackage{paralist}
\usepackage{url}

\theoremstyle{definition}
\newtheorem{question}{Question}
\newtheorem{example}{Example}
\newtheorem{exercise}[question]{Exercise}
\newtheorem*{challenge}{Challenge}
\newtheorem*{theorem}{Theorem}
\newtheorem*{definition}{Definition}

\begin{document}

%Paramètres de mise en forme des paragraphes selon les normes françaises
\setlength{\parskip}{1ex plus 0.5ex minus 0.2ex}
\setlength{\parindent}{0pt}

%Paramètres relatifs aux en-têtes et pieds de page.
\pagestyle{fancy}
\lhead{Theron J Hitchman}
\chead{\Large Reading and Guided Practice \#1}
\rhead{Fall 2013}
\lfoot{\emph{Math and Decision Making}}
\cfoot{}
\rfoot{\emph{\thepage\ of \pageref{LastPage}}}

\section*{Introduction}
We take up the basics of probability. This includes the two styles of language most commonly used to deal with probability questions.

\section*{Goals}
At the end of this assignment, a student should be able to:
\begin{compactitem}
\item Describe the basic situation where the theory of probabilities might be useful.
\item Describe the idea of bias.
\item Make clear statements about probability using the language of propositions.
\item Make clear statements about probability using the language of events.
\item Translate between one language and the other.
\end{compactitem}
A student might also be able to:
\begin{compactitem}
\item Solve a challenging problem about probabilities due to Laplace, one of the founders of probability theory.
\end{compactitem}

\section*{Reading and Questions for 13 November}

\subsection*{The Basic Ideas of Probability}

It is often the case that life presents us with situations involving \emph{chance}.
Many people seek out these situations, say via gambling, for the heightened excitement inherent when things are unpredictable.

Mathematicians and Statisticians have developed ways of thinking about unpredictable situations, and this is called \emph{probability theory}.
There is a big indicator for when the ideas of probability might be useful:
The situation has to involve some true randomness.
Somehow, the variablity in the potential outcomes is not understood, or too complicated to possibly be understood, so things are ``up to chance.''

\subsubsection*{The Idea of Bias}

Sometimes, this manifests itself in the fact that all possible outcomes are known, and all are equally likely.
This is called a \emph{lack of bias}. 
For example, rolling a regular six-sided die is usually considered an unbiased situation.
The key here is that we assume each of the six sides is as likely as any of the others to land up on any particular roll.
If the die is ``loaded'' so that six is much more likely to come up than any other side, then the situation is biased toward six.

It is common in probability theory to assume that the situation is free from bias.
But probability has learned to deal with bias, too, as long as it is not hidden.
If you know in advance that what should be a fair die has been relabeled so that the sides read $1, 2, 3, 4, 6, 6$, then you can still make reasonable arguments, even though the situation is clearly biased toward six.

\subsubsection*{The Basic Mode of Probability}

Probability theory never tries to answer the question of what might happen directly.
It really cannot say, "The die will come up $4$."
That kind of certainty is not available when true randomness is involved.
Instead, probability attempts to describe which outcomes are more or less likely than other outcomes.
The way in which probability information is described is always ``over the long run.''

The idea is that we imagine the situation being presented a great many number of times. 
Each occurance of the situation is called a \emph{trial}.
Probability then tries to describe approximately how many times a certain outcome will arise out of all the trials attempted.

For example, consider again the six-sided die relabelled above.
An individual trial consists of rolling the die once.
Over a very great number of trials (i.e. after rolling the die lots of times) we expect that roughly $\frac{1}{3}$ of the outcomes will be $6$, where only about $\frac{1}{6}$ of the outcomes will be $4$.
So we say that the probability of a $6$ is equal to $\frac{1}{3}$, but the probability of $4$ is $\frac{1}{6}$.

As another example, we discussed the Monty Hall problem in class, and ran simulations.
Each instance of the Monty Hall problem (pick a door, have a door opened and get an offer to switch, then switch-or not) is a trial.
The short computer script we saw ran a million trials.
I have run this little simulation many times, and I can now say with some confidence that it looks like with the switching strategy the probability of winning is about $\frac{2}{3}$. 
This doesn't mean I expect to win two out of every three.
It does mean that over a very large number of attempts, I expect to win about two-thirds of the time.

\subsection*{The Two languages of Probability}

We have already encountered some of the basic terms of probability theory:
\begin{compactdesc}
\item[outcome] A possible final state of the chance event.
\item[trial] One run through of the chance event with a resolution or observation.
\item[bias] A systemic skewing of the results of a chance event towards some set of outcomes and away from others.
\end{compactdesc}

Past these basics, there are two ways to approach probability theory.
These versions are completely equivalent!
The only difference lies in the vocabulary used to describe the setups.

\subsubsection*{The language of Propositions}

In this setup, the chance events are described with simple, declarative sentences, called \emph{propositions}, or \emph{statements}.
Then these propositions are either true or false, and probability of a proposition describes the likelyhood that the statement is true.

For example, the proposition
\begin{quote}
It will rain tomorrow.
\end{quote}
has either the value true or false, but the actual outcome is determined by a complicated set of conditions.
Meteorologists do their best to describe the probability that the statement is true.

Of course, the best they can really say is, ``If we were to have conditions just like today over a sequence of many, many days, then about [whatever fraction] of the time we will get rain the next day.''

An important thing to note is that some sentences don't really fall into the realm of probability.
The sentence must be discernably true or false.
We can wait a day and observe the conditions to see if there is rain or not. 
But some sentences are not so easy.
For example, \textbf{Life is meaningless} is not the kind of thing one can objectively measure or observe and decide to everyone's mutual satisfaction.
Matters of opinion must be left out of probability theory.

The language of propositions is favored by philosophers and logicians, and works naturally for everyday conversations.

\subsubsection*{The Language of Events}

In contrast, the language of events describes probability using the mathematical theory of sets and elements.
One collects all of the possible outcomes into a set called the \emph{sample space}.

Then certain outcomes are collected into subsets, called \emph{events}.

For example, for our unfair die above, the sample space is the set $\{1,2,3,4,6\}$.
The event that a six is rolled is the set $\{6\}$.
The event that the number rolled is even is the set $\{2,4,6\}$.

\begin{exercise}
Take the examples from the last two subsections and tranlate them into the \emph{other} language.
\end{exercise}

We will use the language of events because it is easier to be precise with sets and subsets.
But I think it is important to understand the two languages and how to translate from one to another, because your conversation partner might be more comfortable with different vocabulary.

\subsection*{Challenge: Laplace's Trick Question}

Pierre-Simon Laplace was one of the first people to take up probability as a mathematical study.
He posed the following interesting question, which involves a \emph{compound event}:

You have two urns.
Urn 1 (labeled ``heads'') has 3 red balls in it and 1 green ball in it.
Urn 2 (labeled ``tails'') has 1 red ball in it and 3 green balls in it.

The experiment to be conducted is to flip a coin, and then choose two balls from the corresponding urn, \emph{with replacement}. 
With replacement means that you take a ball out, note its color, and put it back in the urn before drawing again.

\begin{exercise}
Describe the sample space for this situation.
\end{exercise}

\begin{challenge}
What is the probability of the event that one draws two red balls?
\end{challenge}




%\begin{thebibliography}{9}
%\end{thebibliography}

\end{document}
%sagemathcloud={"zoom_width":100}