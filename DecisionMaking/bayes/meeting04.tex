\documentclass[12pt]{amsart}
\usepackage[margin=1in]{geometry}

\theoremstyle{definition}
\newtheorem{task}{Task}

\begin{document}

\begin{center}
\textbf{\Huge
Probability: Class Meeting \#4
}
\end{center}
\vspace{.5in}

It can be challenging to sort out when to use which probability rules. I find it useful to make a simple ``decision tree'' when I feel confused. Try that for these items.

\subsection*{A Die and a Pack of Cards, I}
Suppose you roll a die and choose a card at random from a well-shuffled deck of standard playing cards.

What are the chances that you roll a 4 and choose a heart?\\

What are the chances you roll a 4 or a 2?\\

What are the chances you roll a 4 or choose a heart?\\

\subsection*{A Die and a Pack of Cards, II} Suppose you roll a die and choose a card at random from a well-shuffled deck of playing cards.

What art the chances that you \emph{either} roll a 4 and choose a heart \emph{or} you roll an even number?\\

\subsection*{A Coin and a Die, I} Suppose you flip a coin and roll a die.

What are the chances you get a head or roll one of 2 or 3?\\

\subsection*{A Coin and a Die, II} Suppose you flip a coin. If the coin comes up heads, you roll a six sided die. If the coin comes up tails, you roll a twenty sided die.

What are the chances you get ``a head and a 2"?\\

What are the chances you get a 2, no matter what the coin shows?\\

\subsection*{Synthesize}
So, when do you generally think about adding probabilities? Is there anything you have to watch out for to avoid mistakes?\\

When do you generally think about multiplying probabilities? Is there anything you have to watch out for to avoid mistakes?\\

\clearpage



\end{document}