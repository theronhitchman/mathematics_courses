\documentclass[12pt]{amsart}
\usepackage[margin=1in]{geometry}
\usepackage{paralist}

\theoremstyle{definition}
\newtheorem{question}{Question}

\begin{document}
\begin{center}
\textbf{\Huge
Lesson Plan: Meeting Three
}
\end{center}
\vspace{.5in}

PollEverywhere questions about reversing direction and pulling nails.

\begin{question} Two of these symbols represent the same diagram, just read by traveling the curve in the two different directions. One is a different diagram. Which one is the odd man out?
    \begin{compactitem}
    \item one
    \item two
    \item one read other way
    \end{compactitem}
\end{question}

\begin{question} Consider the diagram with symbol AB*AAB. What symbol do you get from reversing the direction of travel along the curve?
    \begin{compactitem}
    \item BAAB*A
    \item A*BA*A*B*
    \item B*A*A*BA*
    \item BA*BA*A*
    \end{compactitem}
\end{question}

\begin{question} Will the wrapping ABB*A* hold the picture up?  Y/N
\end{question}

\begin{question} 
Consider the attempted solution ABBBA*BBBA*B*B*B*A. Will pulling nail B make the picture fall down? Y/N
\end{question}

\begin{question} 
Consider the attempted solution ABBBA*BBBA*B*B*B*A. Will pulling nail A make the picture fall down? Y/N
\end{question}

\begin{question} How many solutions to the picture hanging puzzle on 2 nails do you know?
    \begin{compactitem}
    \item One, with two descriptions
    \item Two
    \item more than two
    \end{compactitem}
\end{question}



\end{document}