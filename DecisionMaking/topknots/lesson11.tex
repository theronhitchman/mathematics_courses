\documentclass[12pt]{amsart}
\usepackage[margin=1in]{geometry}
\usepackage{paralist}

\theoremstyle{definition}
\newtheorem{question}{Question}

\begin{document}
\begin{center}
\textbf{\Huge
Lesson Plan: Meeting Eleven
}
\end{center}
\vspace{.5in}

\section*{Phase 1}
Discuss idea of links.
Discuss link invariants: number of components, tricolorablity. 

\section*{Phase 2}
Put up five simple links, ask them to differentiate them using our invariants:
\begin{compactenum}
\item hopf link
\item hopf mirror
\item borromean rings
\item borromean rings with one circle swapped out for a trefoil
\item unlink with two components\\
\end{compactenum}

For clarity, ask these specific questions:
\begin{description}
\item[Question 1]  Are any of these distinguishable using our invariants? Which invariant do you use to tell some pairs of them apart?

\item[Question 2] Are any of these links equivalent via ambient isotopy? If so, find a sequence of RMs that realize the equivalence.
\end{description}

\section*{Annoucement}
After class, distribute handout discussing details on using invariants to distinguish between these knots and links.

\end{document}