\documentclass[12pt]{amsart}
\usepackage[margin=1in]{geometry}

\theoremstyle{definition}
\newtheorem{task}{Task}
\newtheorem{question}[task]{Question}

\begin{document}

\begin{center}
\textbf{\Huge
Cantor's Paradise: Class Meeting \#4
}
\end{center}


\vspace{.25in}

Today we will work together toward understanding the central concept for this unit: A way to compare sizes of sets \emph{without counting them first.}
In each case, think about how to describe your ideas so that someone not in this class could understand them.

Most of these activities require you to think about some weird sets.
If the definitions of these sets feel confusing, try first to make some examples and non-examples of elements of the sets involved.
It helps a lot to be talking about the right things.


\begin{question} Without counting, are there more chairs or students in this room?
\end{question}

In mathematics, we often refer to a ``word" as being some string of letters, even if that string is some unreadable mess.
We will need to use this convention!

\begin{question} Let $T$ be the set of 10 letter words, where the first two letters of the word are the same letter.
For example, MMAQRESEDQ is an element of $T$.
Let $J$ be the set of 9 letter words.
For example, PENELOPEJ is an element of $J$.
Which is larger $T$ or $J$?
Or are they the same size?
\end{question}

\begin{task} Let $T$ be the set of all eight digit counting numbers.
Let $J$ be the set of eight letter words using only the letters ABCDEFGHIJ and not beginning with A.
Prove that $T$ and $J$ have the same size.
\end{task}

\begin{task}
Let $T$ be the set of all arrangements of eight / symbols and three * symbols in a row.
These are some different elements of $T$:
\begin{itemize}
\item[] ///*//*//*/
\item[] //*///*//*/
\item[] */////**///
\end{itemize}
Let $J$ be the set of all ways to write $8$ as the sum of four non-negative numbers where order matters.
Here are some different elements of $J$:
\begin{itemize}
\item[] $3+2+2+1=8$
\item[] $2+3+2+1 = 8$
\item[] $0+5+0+3 = 8$
\end{itemize}
Prove that $T$ and $J$ have the same size by finding a matching between the elements of $T$ and the elements of $J$.
\end{task}

\begin{task}
Let $\mathbb{N}$ be the set of all natural numbers, and $\mathcal{E}$ the set of all even natural numbers.
\[
\mathbb{N} = \{ 1, 2, 3, 4, 5, \dots \}, \qquad \qquad \mathcal{E} = \{ 2, 4, 6, 8, 10, \dots \}
\]
Show that $\mathbb{N}$ and $\mathcal{E}$ ``have the same size''.

It is not enough to say ``infinity equals infinity,'' here.
Think about how you have worked through all of the similar problems above.
How can you use the ideas you have been developing?
\end{task}


If this last one doesn't hurt your head a bit, you are not thinking about it correctly.


\end{document}
%sagemathcloud={"zoom_width":105}