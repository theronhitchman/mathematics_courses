\documentclass[12pt,letterpaper]{article}

\usepackage[utf8]{inputenc}
\usepackage[T1]{fontenc}
\usepackage{amsmath}
\usepackage{amsfonts}
\usepackage{amssymb}
\usepackage{amsthm}
\usepackage[left=2cm,right=2cm,top=2cm,bottom=2cm,headheight=22pt]{geometry}
\usepackage{fancyhdr}
\usepackage{setspace}
\usepackage{lastpage}
\usepackage{graphicx}
\usepackage{caption}
\usepackage{subcaption}
\usepackage{paralist}
\usepackage{url}

\theoremstyle{definition}
\newtheorem{question}{Question}
\newtheorem{example}{Example}
\newtheorem{exercise}[question]{Exercise}
\newtheorem*{challenge}{Challenge}
\newtheorem*{theorem}{Theorem}
\newtheorem*{definition}{Definition}

\begin{document}

%Paramètres de mise en forme des paragraphes selon les normes françaises
\setlength{\parskip}{1ex plus 0.5ex minus 0.2ex}
\setlength{\parindent}{0pt}

%Paramètres relatifs aux en-têtes et pieds de page.
\pagestyle{fancy}
\lhead{Theron J Hitchman}
\chead{\Large Reading and Guided Practice \#9}
\rhead{Fall 2013}
\lfoot{\emph{Math and Decision Making}}
\cfoot{}
\rfoot{\emph{\thepage\ of \pageref{LastPage}}}

\section*{Introduction}
In this reading, we explore the existence of numbers which are not rational numbers.
We also introduce the idea of a ``proof by contradiction.''

\section*{Goals}
At the end of this assignment, a student should be able to:
\begin{compactitem}
\item Explain why $\sqrt{2}$ is not a rational number.
\item Explain what a proof by contradiction is.
\end{compactitem}
A student might also be able to:
\begin{compactitem}
\item Give an argument that $\sqrt{3}$ is not a rational number.
\end{compactitem}

\section*{Reading and Questions for 25 October}

We have already encountered many different types of numbers in our study. 
We have seen \emph{natural numbers}, \emph{integers}, and recently \emph{rational numbers}.
There are a great many rational numbers, and they are very useful.
These days, school children spend a lot of time learning about these kinds of numbers, what they might represent, and how to work with them.

\subsection*{An Ancient Discovery}

Of course, at one point in time, detailed knowledge of rational numbers was not so common. 
To the ancient Greek culture mathematicians like Pythagoras and Euclid, dealing with these kinds of quantities was advanced mathematics.
They thought of rational numbers as \emph{ratios}, especially ratios of geometric lengths.

There is an old story, which may or may not be true, that the mystical cult of \emph{Pythagoreans} learned about a new kind of number, which was not expressable as a ratio. 
Much of this group's worldview was built around the idea that ratios were of a singular importance, so this new knowledge was shocking and unsettling.
So unsettling, that the gods drowned the man on a sea voyage. 
You can find a synopsis of the story on Wikipedia\cite{wikipedia}.

For some time, the negative feeling about these quantities remained.
There is still some trace of the negativity in the modern names:
numbers expressable as ratios are called \emph{rational numbers}, numbers which are not expressable as ratios are called \emph{irrational numbers}.

What exactly did the ancient mathematician find?
This poor soul learned that the diagonal of a square with side length equal to $1$ is not expressable as a ratio.
These days, that length is called $\sqrt{2}$, and we instead say this:

\begin{theorem}
The number $\sqrt{2}$ is not a rational number.
\end{theorem}

\subsection*{A Preface on the Argument}

We are going to look at the proof of that theorem.
It is not necessary to be afraid of the word \emph{proof} here.
All that we mean by the word \emph{proof} is that we are going to give a convincing argument about why the theorem is true.

But in this case, the convincing argument has a clever technique in it.
The argument is an example of a \emph{proof by contradiction}. 
What happens is this:
\begin{compactdesc}
\item[Step 1:] We begin by assuming the theorem is \textbf{false}.
\item[Step 2:] We use that assumption to argue that some other thing happens.
\item[Step 3:] The trick is that the ``other thing'' has to be obviously wrong.
\item[Step 4:] Since our assumption led us to something stupid, we conclude that our assumption must be incorrect.
\item[Step 5:] Therefore our theorem must be true.
\end{compactdesc}

This is the logic behind a proof by contradiction.
It can be easy to get things turned around a bit when you are new to this type of argument, because you have to assume negative things a lot.
People who are used to this kind of argument spend all their time on Step 2, and they do not usually explain that the other steps are happening, too.
But after a bit of practice, you will find this natural and all will be well.

\begin{exercise}
Have you ever seen a proof by contradiction before?
Try to think of one or two situations where it has come up.
\end{exercise}

\begin{exercise}
How is a proof by contradiction like a jury trial?
(This is a weak analogy, but there is something to be learned here.)
\end{exercise}

\subsection*{Two Facts we need for the proof}
To make the proof run, we need two facts which you likely already know.
So that things are clearest, we state them right now up front as things you can and should believe.
But so we do not get distracted, we will not give proofs of these statements.

\subsubsection*{The Fundamental Theorem of Arithmetic}

\begin{theorem}
If $b$ is a natural number, then there is a unique way to write $b$ as a product of prime number factors.
\end{theorem}

There are actually two things happening here.
First, every natural number can be written as a product of prime numbers.
For example, $6=2\cdot 3$, and $100 = 2\cdot 2 \cdot 5 \cdot 5$.

\begin{exercise}
Write the number $204$ as a product of prime number factors.
\end{exercise}

Second, there is only one way to write a given number as a product of prime numbers.
The only thing you can do is rearrange the list of primes so it reads in a different order.

For example, the only interesting thing one can do with the expression for $6$ is reorder it like this: $6 = 3\cdot 2$.

\subsubsection*{Rational Numbers in Lowest Terms}

\begin{theorem}
Every rational number $a/b$ can rewritten as an equivalent rational number $c/d$ where $c$ and $d$ have no common prime number factors.
\end{theorem}

The special representation $c/d$ is said to be in \emph{lowest terms.}

\begin{example}
Consider $93/39$.
How may it be expressed in lowest terms?
We factor the numerator and denominator and cancel common factors.
\[
\frac{93}{39} = \frac{3 \cdot 31}{3\cdot 13} = \frac{31}{13}
\]
The equivalent rational number $31/13$ is in lowest terms.
\end{example}

\begin{exercise}
Find a rational number which is equivalent to $465/225$ and is in lowest terms.
\end{exercise}

\subsection*{Proof of the Theorem}

By way of contradiction, assume that the theorem is false.
That is, we assume that $\sqrt{2}$ is a rational number, and therefore expressable as one.
This means that we can choose an integer $a$ and a natural number $b$ so that 
\[
\sqrt{2} = \frac{a}{b}.
\]

We shall make sure to clean things up and choose $a$ and $b$ so that the rational number $a/b$ is in lowest terms.
\textbf{This means that $a$ and $b$ have no prime number factors in common.}

Next, we will do a little algebra.
We multiply through by $b$ to clear the fraction, and then we square both sides of the equation.
Then we have this:
\[ 
2b^2 = a^2.
\]

Now think about the fact that the number on the right and the number on the left are equal.
This means that the expressions of those numbers as products of prime number factors (from the Fundamental Theorem of Arithmetic) must be the same!

From what we see on the left-hand side, it is clear that $2$ appears as a prime number factor in the list.
Therefore, $2$ must be a prime number factor of the right-hand side, that is of $a^2$.
But if $2$ is a prime divisor of $a^2$, it must be a prime number factor of $a$.
From this we conclude that $4 = 2\cdot 2$ is a factor of $a^2$.

Now we go back in the other direction.
Since $4 = 2\cdot 2$ is a factor of $a^2$ it must be a factor of the left-hand side, too.
But the left-hand side is $2 b^2$.
So we must have that $2$ is a factor of $b^2$.
But if $2$ is a prime number factor of $b^2$, it must be a prime number factor of $b$.

\textbf{Therefore, $2$ is a prime number factor of $a$ and $2$ is a prime number factor of $b$.}

This is an absurd situation.
Our two bold statements contradict each other.
Something must be wrong.

What is wrong?
\textit{Our initial assumption!}
We conclude that $\sqrt{2}$ is not a rational number.


\subsection*{Some Challenges}

\begin{challenge}
Adapt the argument above to prove that $\sqrt{3}$ is not a rational number.
\end{challenge}

\begin{challenge}
Adapt the argument above to prove that $\sqrt{6}$ is not a rational number.
\end{challenge}

\begin{challenge}
What happens when you try to adapt the argument to show that $\sqrt{4}$ is not a rational number?
Where does the argument break?
\end{challenge}


\begin{thebibliography}{9}

\bibitem{wikipedia}
    Wikipedia Entry on Hippasus, 
    \url{http://en.wikipedia.org/wiki/Hippasus},
    accessed 24 October, 2013.

\end{thebibliography}

\end{document}
%sagemathcloud={"zoom_width":100}