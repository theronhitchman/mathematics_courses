\documentclass[12pt]{amsart}
\usepackage[margin=1in]{geometry}
\usepackage{paralist}

\theoremstyle{definition}
\newtheorem{task}{Task}

\begin{document}

\begin{center}
\textbf{\Huge
Cantor's Paradise: Class Meeting \#13
}
\end{center}


\vspace{.5in}

Today we take up  the set $\mathcal{R}$ of all real numbers between $0$ and $1$.
We are going to think about them in decimal notation, where each element will be represented by a string of digits which is infinite in length, like $0.112343145213\ldots$

We begin by trying to show that $\mathcal{R}$ is countably infinite.
Don't worry, I am not going to ask you to come up with a matching between the elements of $\mathcal{R}$ and the elements of $\mathbb{N}$!
What we will do is pretend we have such a matching, and see what that might tell us.

\begin{task}
Suppose that we have a matching between the elements of $\mathcal{R}$ and the elements of $\mathbb{N}$.
As we have seen with other examples of countably infinite sets, this matching will be given by an ordered list that hits each element of $\mathcal{R}$ exactly once.
Suppose that the list \textbf{starts} like this:
\begin{align*}
1 & \leftrightarrow  0.3374543377733 \dots \\
2 & \leftrightarrow  0.3733333777777 \dots \\
3 & \leftrightarrow  0.2335247700809 \dots \\
4 & \leftrightarrow  0.7878709870777 \dots \\
5 & \leftrightarrow  0.1010000010011 \dots \\
6 & \leftrightarrow  0.0000030000040 \dots \\
\vdots & \ \ \vdots  \qquad \ddots
\end{align*}
This list goes on and on to pair a real number with each element of $\mathbb{N}$.

We want to make a \emph{new real number}, $A$, by specifying its decimal representation.
Here is how: to get the $n$th digit of $A$, look at the $n$th digit past the decimal point of the $n$th item in our list.
If that digit is a $3$, we make the $n$th digit of $A$ a $7$; if that digit is not a $3$, then we make the $n$th digit of $A$ a $3$.

Write out the first six digits of $A$.

\end{task}

\begin{task}
\begin{compactitem}
\item Explain why $A$ is not equal to the first item on the list.
\item Explain why $A$ is not equal to the second item on the list.
\item Explain why $A$ is not equal to the third item on the list.
\end{compactitem}
\end{task}


\begin{task}
Write a sentence or two that describes why $A$ cannot appear on our list.
\end{task}

OK. That list was not the best. Let's try another.

\begin{task}
Each member of your group should come up with a beginning of a new list which is supposed to make a matching between $\mathcal{R}$ and $\mathbb{N}$.
Six elements of $\mathcal{R}$ is enough.
\end{task}

\begin{task}
Trade papers so that everyone in the group has someone else's list.
Use the recipe above to come up with an element $A'$ and explain why this new $A'$ is not on the list your partner started.
\end{task}

Think for a minute about what has happened so far.
What does this all mean?

\begin{task}
Give an argument by contradiction that $\mathcal{R}$ is \textbf{not} countably infinite.
\end{task}



\end{document}
%sagemathcloud={"zoom_width":75}