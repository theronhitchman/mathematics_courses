\documentclass[12pt,letterpaper]{article}

\usepackage[utf8]{inputenc}
\usepackage[T1]{fontenc}
\usepackage{amsmath}
\usepackage{amsfonts}
\usepackage{amssymb}
\usepackage{amsthm}
\usepackage[left=2cm,right=2cm,top=2cm,bottom=2cm,headheight=22pt]{geometry}
\usepackage{fancyhdr}
\usepackage{setspace}
\usepackage{lastpage}
\usepackage{graphicx}
\usepackage{caption}
\usepackage{subcaption}
\usepackage{paralist}

\theoremstyle{definition}
\newtheorem{question}{Question}
\newtheorem{example}{Example}
\newtheorem{exercise}[question]{Exercise}
\newtheorem*{challenge}{Challenge}
\newtheorem*{theorem}{Theorem}
\newtheorem*{problem}{Problem}

\begin{document}

%Paramètres de mise en forme des paragraphes selon les normes françaises
\setlength{\parskip}{1ex plus 0.5ex minus 0.2ex}
\setlength{\parindent}{0pt}

%Paramètres relatifs aux en-têtes et pieds de page.
\pagestyle{fancy}
\lhead{Theron J Hitchman}
\chead{\Large Reading and Guided Practice \#6}
\rhead{Fall 2013}
\lfoot{\emph{Math and Decision Making}}
\cfoot{}
\rfoot{\emph{\thepage\ of \pageref{LastPage}}}

\section*{Introduction}
In this reading, you will learn about Reidemeister moves and their significance for knot theory.

\section*{Goals}
At the end of this assignment, a student should be able to:
\begin{compactitem}
\item Identify and use the three Reidemeister moves when discussing knot equivalence.
\item Show two knots are equivalent by ambient isotopy by exhibiting a sequence of Reidemester moves.
\end{compactitem}
A student might also be able to:
\begin{compactitem}
\item Solve a challenging problem using Reidemeister moves.
\end{compactitem}

\section*{Reading and Questions for 11 September}

\subsection*{Context and History for the Reidemeister moves}
We have seen that the natural way to view two knots as equivalent is the notion of \emph{ambient isotopy}. We have also played around enough to see that this is very tricky. There is no way to collect ``all of the ambient isotopies possible'' and make sure we have checked them---there are simply too many strange things one can do.

Fortunately, there is a way around this, first described by J.~W.~Alexander and G.~B.~Briggs \cite{AB} and independently by H.~Reidemeister \cite{R}. There are some simple manipulations one can do to a planar projection of a knot, called \emph{Reidemeister moves}. They are important because of the following result, which was proved in the 1920's.
\begin{theorem}
Suppose that one is given two planar projections of knots. Then these knots are equivalent by ambient isotopy if and only if there is a finite sequence of \emph{Reidemeister moves} which turns one planar projection into the other.
\end{theorem}

This is exceedingly good news! You don't have to check ``everything,'' just sequences of these special moves on knots.  

The general problem for comparing two knots is still unsolved, but significant work has been done on the ``unknotting problem.''
\begin{problem}
When and how can you tell if a given planar projection of a knot actually represents the unknot?
\end{problem}
It is known that there is an algorithm for deciding this question, but the paper describing it  (by Haken \cite{Haken}) is well over 100 pages long. In principle, this means someone could program a computer to check if a given diagram was an unknot or not. [I do not believe anyone has done this.] One problem that could occur is that even if such a computer program could be designed, it might take an unreasonably long time to compute and give you a response. How long might it take? In 1998, Joel Hass and Jeffrey Lagarias proved that it might be long, but not \emph{too} long. Specifically, they showed this:
\begin{theorem}
For any planar projection of the unknot having $n$ crossings, there is a sequence of Reidemeister moves which transforms the projection into a regular circle and takes no more than $2^{10^11}n$ moves.
\end{theorem}

That is a big number, but computers are pretty fast.

\subsection*{The Reidemeister Moves}

So what are the Reidemeister moves? Each move describes a way in which you can change some small bit of information about a crossing in your planar projection without changing the essential qualities of the knot. There are three types of moves, and they are easier to see than to describe.

[[insert moves pictures]]


\subsubsection*{Examples and Practice} 

Let us see how to handle some examples of equivalence of planar projections using Reidemeister moves and isotopy.

\begin{example} 
A simple trefoil with a cheap disguise.
We just untwist.
\end{example}

[[insert picture]]

\begin{example}
A trefoil with a better disguise.
Here we bring the far right strand ``under the rest of the diagram'' and then untwist.
\end{example}

[[insert picture]]


Now practice.

\begin{exercise} Consider the two planar projections of the figure eight knot below. 
These are equivalent by a process similar to our second example above.
Use a sequence of Reidemeister moves to demonstrate that these knots are equivalent. 
Draw each intermediate knot, and label the sequence of Reidemeister moves you use as you go.
\end{exercise}

[[insert picture]]


\begin{exercise} Consider the two planar projections of the knot below. 
Use a sequence of Reidemeister moves to demonstrate that these knots are equivalent. 
Draw each intermediate knot, and label the sequence of Reidemeister moves you use as you go.
\end{exercise}

[[insert picture]]


\begin{challenge}
The following knot is due to Goeritz. It is known to be a projection of the unknot and it has 9 crossings. 
Use a sequence of Reidemeister moves to show that this is the unknot.
Somewhere along the line, your diagram should have more than 9 crossings.
\end{challenge}


%\begin{thebibliography}{9}


%\bibitem{Demaine}
%    Erik Demaine, Martin Demaine, Yair Minsky, Joseph Mitchell, Ronald Rivest, \& Mihai Patrascu,
%    Picture Hanging Puzzles,
%    available online at the arXiv: 1203.3602v1,
%    16 March 2012

%\end{thebibliography}

\end{document}
%sagemathcloud={"zoom_width":100}