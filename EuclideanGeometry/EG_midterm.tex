\documentclass[11pt]{amsart}

\usepackage[margin=1in]{geometry}

\theoremstyle{definition}
\newtheorem{task}{Task}


\begin{document}

\title{Euclidean Geometry\\ Midterm Exam}
\author{18 October 2013}
\maketitle


\noindent
\textbf{Instructions:} Please use complete sentences and explain yourself as clearly as you can.
You may use your notes and your copy of the \emph{Elements}, but I think you are better off just using your brain.
It is not necessary to quote results by their proper numbers, but results you use should be stated and used properly.
As usual, any item from the \emph{Elements} I.1-34 and any item proved in class may be used in your work.

This is a one hour exam.
You are on your honor to spend one hour on this exam, though you may do so at a time and in a place of your choosing.
When you have completed an hour of work, please place the examination paper and your pages with solutions in the envelope provided and hand them in at the department office.



\begin{task}
Prove that the diagonals of a parallelogram bisect each other.
\end{task}

\begin{task}\label{contradiction}
Recall that we have already proved the following theorem:
\begin{quote}
\textbf{Theorem 3.6:} (Sawada) Let $ABC$ be a triangle with $D$ the midpoint of $AB$ and $E$ the midpoint of $AC$.
The line $DE$ is parallel to the line $BC$.
\end{quote}
I am interested in a new proof of this theorem that uses an argument by contradiction.
Write out the beginning of such a proof.
You need not complete the proof, but you should begin it well.
\end{task}

\begin{task}\label{biconditional}
We have considered several ways to express ideas by making new definitions.
Often, there is more than one way to make a technical condition encompassing an idea.
To reduce confusion, it becomes important to show that these conditions are equivalent.

In class, we made this definition:
\begin{quote}
\textbf{Definition:} Let $ABC$ be a triangle, and let $X$ be a point. We shall say that $X$ lies \emph{inside} triangle $ABC$ when the six angles $XAB$, $ABX$, $XBC$, $BCX$, $XCA$, and $CAX$ taken together make two right angles.
\end{quote}
A way to state a theorem saying this formulation is equivalent to a different one looks like this:
\begin{quote}
\textbf{Theorem:} Let $ABC$ be a triangle and $X$ a point. Then $X$ lies inside triangle $ABC$ if and only if $A$ and $X$ lie on the same side of line $BC$, $B$ and $X$ lie on the same side of line $AC$, and $C$ and $X$ lie on the same side of line $AB$.
\end{quote}

Without actually giving a proof, explain how you would structure a proof of this theorem.
That is, do not write a complete proof of this result, but give a high-level outline of what would be necessary in an argument for this result.
\end{task}

\begin{task}
Let $ABC$ be an equilateral triangle. We wish to choose a point $D$ in the plane so as to construct a quadrilateral $ABCD$.
(This is just a traditional name for a $4$-gon.)
Where can we put this new point?

Draw a diagram and label the places where $D$ cannot be placed to make a quadrilateral.
For each part of your figure where $D$ cannot be placed, label it with the reason that $D$ cannot be placed there.
Also, for those parts of your diagram where one may place $D$, label it with any interesting description of the shape of the resulting quadrilateral that might be relevant.
\end{task}



\end{document}





























%sagemathcloud={"zoom_width":100}